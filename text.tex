% easychair.tex,v 3.5 2017/03/15

\documentclass{easychair}
%\documentclass[EPiC]{easychair}
%\documentclass[EPiCempty]{easychair}
%\documentclass[debug]{easychair}
%\documentclass[verbose]{easychair}
%\documentclass[notimes]{easychair}
%\documentclass[withtimes]{easychair}
%\documentclass[a4paper]{easychair}
%\documentclass[letterpaper]{easychair}

 
\input{../../git/common/preamble}
\input{../../git/common/math}
\tikzstyle{class}=[rectangle split,rectangle split parts=2,draw,align=left]
\tikzstyle{classWithOp}=[rectangle split,rectangle split parts=3,draw,align=left]
\tikzstyle{processStep} = [signal,draw,align=center,signal from=west, signal to=east]
\tikzstyle{dot} = [circle,fill=black,scale=0.8]
\tikzstyle{database} = [cylinder,draw,shape border rotate=90,aspect=0.25]
\tikzstyle{mycloud} = [cloud,draw, cloud puffs=10,cloud puff arc=120,aspect=2, inner ysep=1em]

\tikzstyle{bpmnTask} = [rectangle, draw, black,
minimum width=4em, minimum height=2em,rounded corners,align=center,every task]
\tikzstyle{bpmnGateway} = [diamond, draw, black,inner sep=0pt,minimum width=2em, minimum height=2em,every gateway]
\tikzstyle{bpmnSequence} = [->,>=triangle 45,every sequence]
\tikzstyle{bpmnmessage} = [o->,dashed,>=open triangle 45,every sequence]
\tikzstyle{bpmnAssociation} = [->,densely dotted,>=angle 45,every association]
\tikzstyle{bpmnEvent} = [circle,minimum width=1.5em, minimum height=1.5em,draw,every event]
\tikzstyle{bpmnEndEvent} = [event,ultra thick,every event]
\tikzstyle{bpmnIntermediateEvent} = [event,double,every event]
\colorlet{punct}{red!60!black}
\definecolor{delim}{RGB}{20,105,176}
\colorlet{numb}{magenta!60!black}
\definecolor{gray}{rgb}{0.4,0.4,0.4}
\definecolor{darkblue}{rgb}{0.0,0.0,0.6}
\definecolor{lightblue}{rgb}{0.0,0.0,0.4}
\definecolor{deepmagenta}{rgb}{0.8, 0.0, 0.8}

\lstdefinelanguage{json}{
	basicstyle=\tiny\normalfont\ttfamily,
	numbers=none,
	numberstyle=\scriptsize, 
	stepnumber=1,
	numbersep=8pt,
	showstringspaces=false,
	breaklines=true,
	frame=single,
	literate=
	*{0}{{{\color{numb}0}}}{1}
	{1}{{{\color{numb}1}}}{1}
	{2}{{{\color{numb}2}}}{1}
	{3}{{{\color{numb}3}}}{1}
	{4}{{{\color{numb}4}}}{1}
	{5}{{{\color{numb}5}}}{1}
	{6}{{{\color{numb}6}}}{1}
	{7}{{{\color{numb}7}}}{1}
	{8}{{{\color{numb}8}}}{1}
	{9}{{{\color{numb}9}}}{1}
	{:}{{{\color{punct}{:}}}}{1}
	{,}{{{\color{punct}{,}}}}{1}
	{\{}{{{\color{delim}{\{}}}}{1}
	{\}}{{{\color{delim}{\}}}}}{1}
	{[}{{{\color{delim}{[}}}}{1}
	{]}{{{\color{delim}{]}}}}{1},
}

\definecolor{maroon}{rgb}{0.5,0,0}
\definecolor{darkgreen}{rgb}{0,0.5,0}
\lstdefinelanguage{customxml}{
	basicstyle=\tiny\normalfont\ttfamily,
	numbers=none,
	numberstyle=\scriptsize, 
	stepnumber=1,
	numbersep=8pt,
	showstringspaces=false,
	frame=single,
morestring=[s]{"}{"},
morecomment=[s]{?}{?},
morecomment=[s]{!--}{--},
commentstyle=\color{darkgreen},
moredelim=[s][\color{black}]{>}{<},
moredelim=[s][\color{red}]{\ }{=},
stringstyle=\color{blue},
identifierstyle=\color{maroon}
}

\lstdefinelanguage{interrop}{
	basicstyle=\small\scriptsize\ttfamily,
	numbers=left,
	numberstyle=\tiny, 
	morekeywords={correspondence,as,relate,check,with,exists,all,and,or,constraint,correspondences,endpoints,where,equals,fields, identify,concat},
	keywordstyle=\color{maroon},
	stringstyle=\color{darkblue},
	moredelim={[s][stringstyle]{"}{"}}
}

\lstdefinelanguage{sigma}{
	basicstyle=\small\ttfamily,
	numbers=none,
	numberstyle=\scriptsize, 
	morekeywords={sig,sorts,opns, eqns, import},
	keywordstyle=\bfseries
}

\lstdefinelanguage{GQL}{
	basicstyle=\scriptsize\ttfamily,
	morekeywords={type,extend},
	keywordstyle=\color{maroon},
	emph={Int,String,Boolean,ID,Float, Query,Mutation},
	emphstyle=\color{darkblue},
	stringstyle=\color{deepmagenta},
	moredelim={[s][stringstyle]{@}{)}}
}
\lstdefinelanguage{js}{
	basicstyle=\small\scriptsize\ttfamily,
	numbers=left,
	numberstyle=\scriptsize, 
	morekeywords={function,var,const,return, if, else, while, for, do, break, continue, switch,default,typeOf,void},
	keywordstyle=\color{maroon}
}

\lstdefinelanguage{BNF}
{
	showstringspaces=false,
	basicstyle=\small,
	morestring=[b]",
	morestring=[s]{>}{<},
	morecomment=[s]{<?}{?>},
	stringstyle=\color{black},
	identifierstyle=\color{darkblue},
	keywordstyle=\color{darkblue},
	morekeywords={xmlns,targetNamespace,name, xmlns:soap, xmlns:tns, xmlns:xsd, type, message, style, transport, soapAction, encodingStyle, namespace, use, binding, location}
}

\usepackage{tikzit}
\input{../../git/common/drawing.tikzstyles}

% The categories we are dealing with
\newcommand{\catSET}{\mathbb{S}\mathbbm{et}}
\newcommand{\catINCL}{\mathbb{I}\mathbbm{ncl}}
\newcommand{\catCAT}{\mathbb{C}\mathbbm{at}}
\newcommand{\catPAR}{\mathbb{P}\mathbbm{ar}}
\newcommand{\catREL}{\mathbb{R}\mathbbm{el}}
\newcommand{\catSEN}{\mathbb{S}\mathbbm{en}}
\newcommand{\catMULT}{\mathbb{M}\mathbbm{ult}}
\newcommand{\catMON}{\mathbb{M}\mathbbm{on}}
\newcommand{\catGRAPH}{\mathbb{G}\mathbbm{raph}}
\newcommand{\catSIMPLEGRAPH}{\mathbb{SG}\mathbbm{raph}}
\newcommand{\catHYPERGRAPH}{\mathbb{H}\mathbbm{yp}}
\newcommand{\catALGOMEGA}{\mathbb{A}\mathbbm{lg}(\Omega)}
\newcommand{\catALGSIGMA}{\mathbb{A}\mathbbm{lg}(\Sigma)}
\newcommand{\catSYSSIGMA}{\mathbb{S}\mathbbm{ys}(\Sigma)}
\newcommand{\catExplanation}[4]{
\begin{description}[leftmargin=\parindent,labelindent=\parindent,itemsep=2pt,parsep=2pt]
\item[Objects] {#1}
\item[Morphisms] {#2}
\item[Composition] {#3}
\item[Identities] {#4}
\end{description}
}
%% Front Matter
%%
% Regular title as in the article class.
%
\title{Category Theory Compendium}


% Authors are joined by \and. Their affiliations are given by \inst, which indexes
% into the list defined using \institute
%
\author{
	Patrick Stünkel\inst{1}%\thanks{Designed and implemented the class style}
%	\and
%	Somebody else\inst{2}%\thanks{Did numerous tests and provided a lot of suggestions}
}

% Institutes for affiliations are also joined by \and,
\institute{
	Western Norway University of Applied Sciences,
	Bergen, Norway\\
	\email{past@hvl.no}
%	\and
%	Other University,
%	Other, Country\\
%	\email{other@university.edu}\\
}

%  \authorrunning{} has to be set for the shorter version of the authors' names;
% otherwise a warning will be rendered in the running heads. When processed by
% EasyChair, this command is mandatory: a document without \authorrunning
% will be rejected by EasyChair

\authorrunning{P. Stünkel}

% \titlerunning{} has to be set to either the main title or its shorter
% version for the running heads. When processed by
% EasyChair, this command is mandatory: a document without \titlerunning
% will be rejected by EasyChair
\titlerunning{Category Theory Compendium}

\begin{document}

\maketitle

\begin{abstract}
All the proofs I know
\end{abstract}


% The table of contents below is added for your convenience. Please do not use
% the table of contents if you are preparing your paper for publication in the
% EPiC Series or Kalpa Publications series


%\section{To mention}
%
%Processing in EasyChair - number of pages.
%
%Examples of how EasyChair processes papers. Caveats (replacement of EC
%class, errors).

%------------------------------------------------------------------------------
\section{Categories}
\label{sec:introduction}

\input{section/foundatios}

\section{Morphisms}
\label{sec:morphisms}


\subsection{Isomorphisms}

\mkDefinition{Isomorphisms}{def:isos}{
	A morphism $\arrow{A}{f}{B}$ is called an \emph{isomorphism} if there exists a morphism $\arrow{B}{g}{A}$ such that 
	\begin{equation}
	\label{eq:iso}
	\tag{Invertability}
	\compose{f}{g} = \identity{A} \wedge \compose{g}{f} = \identity{B}
	\end{equation}
	Two objects $A, A'$ are called \emph{isomorphic}, denoted $A \isomorphic A'$, if there exists an isomorphism $\arrow{A}{i}{A'}$.
}

\mkProposition{Identities are Isomorphisms}{prop:idsAsIsos}{
Every identity $\arrow{A}{\identity{A}}{A}$ is an isomorphisms.
}
\begin{proof}
	Define $\inverse{\identity{A}} := \identity{A}$.
	Evidently, \ref{eq:section} and \ref{eq:section} are immediately fulfilled because of the \ref{eq:neutrality}.
\end{proof}


\mkDefinition{Sections and Retractions}{def:sectionsRetractions}{
Let $\arrow{A}{f}{B}$ be an arrow 
\begin{itemize}
\item A \emph{section} for $f$ is a morphism $\arrow{B}{s}{A}$ such that: 
\begin{equation}
\tag{Section}
\label{eq:section}
\compose{s}{f}  = id_B
\end{equation}
\item A \emph{retraction} for $f$ is a morphism $\arrow{B}{r}{A}$ such that
\begin{equation}
\tag{Retraction}
\label{eq:retraction}
\compose{f}{r} = id_A
\end{equation}
\end{itemize}
}


\mkProposition{Isomorphisms, Sections and Retractions compose}{prop:isoComposition}{
If $\arrow{A}{f}{B}$ is an isomorphism and $\arrow{B}{g}{C}$ is an isomorphism too, then their compositions $g \circ f$ is an isomorphism too. Likewise for sections and retractions.
}
\begin{proof}
\begin{equation}
\label{diag:isoComposition}
\xymatrix{
A \ar@/^/[r]^f & B \ar@/^/[r]^g \ar@{-->}@/^/[l]^{f^{-1}} & C  \ar@{-->}@/^/[l]^{g^{-1}}
}
\end{equation}
Assume $f$ and $g$ are sections, then there are morphisms $\arrow{B}{\inverse{f}}{A}$ and $\arrow{C}{\inverse{g}}{B}$ (LI-f), depicted in diagram~\ref{diag:isoComposition} such that $\compose{f}{\inverse{f}} = id_A$ and $\compose{g}{\inverse{g}} = id_B$ (LI-g), cf. \ref{eq:section}. 
In deed, $\inverse{(\compose{f}{g})} := \compose{\inverse{g}}{\inverse{f}}$ is left inverse of $\compose{f}{g}$ because
\begin{align*}
\compose{(\compose{f}{g})}{(\compose{\inverse{g}}{\inverse{f}})}  & \\
\proofStep{\ref{eq:associativity}} & = \compose{\compose{f}{(\compose{g}{\inverse{g}})}}{\inverse{f}}\\ 
\proofStep{LI-g}  & =   \compose{\compose{f}{\identity{B}}}{\inverse{f}} \\ 
\proofStep{\ref{eq:neutrality}} & =  \compose{f}{\inverse{f}} \\ 
\proofStep{LI-f} & = \identity{A}
\end{align*}
Likewise, assuming $f$ and $g$ to be retractions, there exist morphisms $\arrow{B}{\inverse{f}}{A}$ and $\arrow{C}{\inverse{g}}{B}$, such that $\compose{\inverse{f}}{f} = \identity{B}$ (RI-f) and $\compose{\inverse{g}}{g} = \identity{C}$ (RI-g), cf. \ref{eq:retraction} and we dually get
\begin{align*}
\compose{(\compose{\inverse{g}}{\inverse{f}})}{(\compose{f}{g})} & \\
\proofStep{\ref{eq:associativity}}	& = \compose{\inverse{g}}{\compose{(\compose{\inverse{f}}{f})}{f}} \\ 
\proofStep{RI-f} & =  \compose{\inverse{g}}{\compose{\identity{B}}{g}} \\ 
\proofStep{\ref{eq:neutrality}} \rangle & =  \compose{\inverse{g}}{g}  \\ 
\langle \mbox{RI-g} \rangle& = \identity{C}
\end{align*}
Both statements together conclude the proof for composition of isomorphisms.
\end{proof}

\mkProposition{Inverses are unique}{prop:uniqueInverse}{
When $\arrow{A}{f}{G}$ is an isomorphism its inverse is determined \emph{uniquely}.
}
\begin{proof}
Let $\arrow{A}{f}{B}$ be a morphism and $\arrow{B}{g,h}{A}$ morphism for which \ref{eq:section} and \ref{eq:retraction} holds, i.e. $\compose{f}{g} = \identity{A}$ (i), $\compose{f}{h} = \identity{A}$ (ii), $\compose{g}{f} = \identity{B}$ (iii) and $\compose{h}{f} = \identity{B}$ (iv).
Thus
\begin{align*}
	g & \\
	\proofStep{\ref{eq:neutrality}} & = \compose{g}{\identity{A}}\\
	 \proofStep{(ii)} & = \compose{g}{(\compose{f}{h})}  \\
	 \proofStep{\ref{eq:associativity}} & = \compose{(\compose{g}{f})}{h} \\
	 \proofStep{(ii)} & = \compose{\identity{B}}{h} \\
	 \proofStep{\ref{eq:neutrality}} & = h
\end{align*} 
\end{proof}
This justifies the notion $\inverse{f}$ that is both a left and right inverse for a morphism $f$.	

\mkProposition{Inverses are isomorphisms}{prop:inverseIsomorphisms}{
Let $\arrow{A}{f}{B}$ be an isomorphism than the morphism $\arrow{B}{\inverse{f}}{A}$ that satisfies both \ref{eq:section} and \ref{eq:retraction} is an isomorphism too.
}
\begin{proof}
We define the inverse of $\inverse{f}$ as $\inverse{(\inverse{f})} := f$.
This evidently fulfills the requirement since $f$ is an isomorphism.
\end{proof}

\mkProposition{Isomorphism Equivalence Relation}{prop:isoEqui}{
The \emph{isomorphism} relation $\isomorphic \subseteq \catObjects{\cat{C}} \times \catObjects{\cat{C}}$ over objects in a category $\cat{C}$ is an equivalence relation.
}
\begin{proof}
For $A, B, C \in \catObjects{\cat{C}}$, \emph{reflexivity} $A \isomorphic A$ follows from Prop.~\ref{prop:idsAsIsos}, \emph{symmetry} $A \isomorphic B \Rightarrow B \isomorphic A$ follows from Prop.~\ref{prop:inverseIsomorphisms}, and \emph{transitivity} $A \isomorphic B \wedge B \isomorphic C \Rightarrow A \isomorphic C$ follows from Prop.~\ref{prop:isoComposition}.
\end{proof}

\mkDefinition{Determination and Choice}{def:determinationChoice}{
	\ \par
\begin{enumerate}
\item Given two arrows $\arrow{A}{f}{B}$ and $\arrow{A}{h}{C}$, finding an arrow $\arrow{B}{g}{C}$ such that $\compose{f}{g} = h$ is called the \emph{determination} or \emph{extension} problem.
\begin{equation}
\label{eq:determination}
\tag{Determination}
\xymatrix{
 & B \ar@{-->}[dr]^{g?} & \\
 A \ar[ur]^{f} \ar[rr]_{h} & & C
}
\end{equation}
\item Given two arrows $\arrow{B}{g}{C}$ and $\arrow{A}{h}{C}$, finding an arrow $\arrow{A}{f}{B}$ such that $\compose{f}{g} = h$ is called the \emph{choice} or \emph{lifting} problem.
\begin{equation}
\label{eq:choice}
\tag{Choice}
\xymatrix{
	& B \ar[dr]^g & \\
	A \ar@{-->}[ur]^{f?} \ar[rr]_{h} & & C
}
\end{equation}
\end{enumerate}
}

\mkProposition{Sections as Choice Problem}{prop:sectionsAsChoices}{
If for an arrow $\arrow{A}{f}{B}$, the choice problem has a solution for every $C$ and $\arrow{C}{h}{B}$, then $f$ has a section.
}
\begin{proof}
The section $\arrow{B}{s}{A}$ is given as the solution of the following choice problem
\begin{equation}
\xymatrix{
 & A \ar[dr]^f & \\
B \ar[rr]_{\identity{B}} \ar@{-->}[ur]^{s} & & B 
}
\end{equation}
\end{proof}

\mkProposition{Sections solve Choice Problem}{prop:sectionSolvesChoices}{
If an arrow $\arrow{A}{f}{B}$ has a section, then for any object $C$ and arrow $\arrow{C}{h}{B}$ than the choice problem has a solution, i.e. there exists an arrow $\arrow{C}{g}{A}$ such that $\compose{g}{g} = h$.
}
% TODO proof


\begin{myprop}[Section decomposition]
If $f;g$ is a section, then $f$ is a section as well.
\end{myprop}

\begin{myprop}[Retraction decomposition]
	If $f;g$ is a retraction, then $g$ is a retraction as well.
\end{myprop}

\begin{myfact}[Sections in $\catSET$]
Sections in $\catSET$ is exactly the class of \emph{injective} mappings.
\end{myfact}


% TODO determinism and choice problems.

\subsection{Monomorphisms}


\begin{mydef}{Monomorphism}
A morphism $\arrow{B}{m}{C}$ is called  \emph{monic} or a \emph{monomorphism} iff. $\forall \arrow{A}{f,g}{B}$ the following implication holds $m \circ f = m \circ g \implies f = g$.
A monomorphism $m$ is called
\begin{description}
\item[regular] iff it is an equalizer of a diagram.
\item[extremal] iff for every factorization $m = f \circ e$ where $e$ is epic, $e$ must be an isomorphism.
\end{description}
\end{mydef}

\begin{myprop}[Monic decomposition]
	If $f;g$ is monic, then $f$ is monic as well.
\end{myprop}

\begin{myprop}[Monic composition]
	If $f;$  and $g$ are monic, then $f;g$ is monic as well.
\end{myprop}

\begin{myprop}[Sections are monic]
Every section $f$ is a monomorphism.
\end{myprop}

\begin{myprop}[Monic retractions are isomorphisms]
If $f$ is a monomorphism and a retraction, the $f$ is an isomorphism.
\end{myprop}

\begin{myfact}[Retractions in $\catSET$]
	Sections in $\catSET$ is exactly the class of \emph{surjective} mappings.
\end{myfact}

\begin{myfact}[Isomorphisms in $\catSET$]
	Isomorphisms in $\catSET$ is exactly the class of \emph{bijective} mappings.
\end{myfact}

\begin{myfact}[Monomorphisms and Epimorphisms $\catINCL$]
	All morphisms in $\catINCL$ are monomorphisms and epimorphisms.
\end{myfact}

\begin{myfact}[Monomorphisms in $\catPAR$]
	A morphism $f: A \rightharpoonup B$ is a monomorphism iff it is total and injective.
\end{myfact}

\begin{myfact}[Monomorphisms in $\catPAR$]
	A morphism $f: A \rightharpoonup B$ is a monomorphism iff it is surjective.
\end{myfact}




\section{Factorization Systems}
\label{sec:factorizationSystems}
\input{sections/factorization}

\section{Functors}
\label{sec:functors}

\begin{myprop}[Functors preserve isomorphisms]
\end{myprop}


% TODO Def FUNCTOR CATS






\section{Universal Constructions}
\label{sec:universalConstructions}


\subsection{Terminal and Initial Objects}
Final and initial objects are defined over an \emph{empty} diagram, i.e. they relate uniquely to \emph{all} objects in the category.


\begin{mydef}[Final and Initial objetc]
	An object $1 \in \cat{C}$ is said to be \emph{final} iff. for all $X \in \cat{C}$ there exists a unique morphism $\arrow{X}{1_X}{1}$.
	Dually, an object $0 \in \cat{C}$ is said to be \emph{initial} iff for all $X \in \cat{C}$ there exists a unique morphism $\arrow{0}{0_X}{C}$.
	\[
	\xymatrix{
		X \ar@{.>}[d]^{1_X} & & 0 \ar@{.>}[d]^{0_X}  \\
		1  & & X
	}
	\]
\end{mydef}

\begin{myprop}[Final object in $\catSET$]
	A final object in $\catSET$ is a one-element set $1 = \{\bullet\}$.
\end{myprop}

\begin{myprop}[Initial object in $\catSET$]
	The initial object in $\catSET$ is the empty set $0 = \emptyset$.
\end{myprop}

\begin{myprop}[Initial object in $\catINCL$]
	The initial object in $\catINCL$ is the empty set $0 = \emptyset$.
\end{myprop}

\begin{myprop}[Initial object in $\catINCL$]
	There is not terminal object in $\catINCL$.
\end{myprop}

\begin{myprop}[Initial object in $\catPAR$]
	The terminal object in $\catPAR$ is the empty set $1 = \emptyset$.
\end{myprop}

\begin{myprop}[Initial object in $\catREL$]
	The terminal object in $\catREL$ is the empty set $0 = \emptyset$.
\end{myprop}



\subsection{Products and Coproducts}

\begin{mydef}[Products and Coproducts]
	Let $\cat{C}$ be a category and $A, B \in \cat{C}$ be objects.
	A binary \emph{product} of $A$ and $B$ written $A \times B$ is given by a span $(A\times B, \arrow{A \times B}{\pi_A}{A}, \arrow{A \times B}{\pi_B}{B})$ such that for all spans $(X,\arrow{X}{f}{A},\arrow{X}{g}{B})$ there is a unique morphism $\arrow{X}{(f,g)}{A\times B}$ such that ...
	\[
	\xymatrix{
		& C \ar@/_1pc/@{-->}[ddl]_{f} \ar\ar@/^1pc/@{-->}[ddr]^{g} \ar@{.>}[d]|{\langle f,g\rangle} &  & & & C & \\
		& A \times B \ar[dl]_{\pi_A} \ar[dr]^{\pi_B} & & &  & A + B \ar@{.>}[u]|{[f,g]} &  \\
		A & & B & & A \ar[ur]^{\kappa_A} \ar@/^1pc/@{-->}[uur]^{f} & & B \ar[ul]_{\kappa_B} \ar@/_1pc/@{-->}[uul]_{g}
	}
	\]
\end{mydef}

\begin{myprop}[Products are asssociative]
	$(A \times B) \times C \cong A \times (B \times C)$
\end{myprop}
\begin{myprop}[Products are commutative]
	$A \times B \cong B \times A$
\end{myprop}
\begin{myprop}[Products are neutral wrt Final Objects]
	$A \times 1 \cong A$
\end{myprop}

\begin{myprop}[Coproducts are asssociative]
	$(A + B) + C \cong A + (B + C)$
\end{myprop}
\begin{myprop}[Coproducts are commutative]
	$A + B \cong B \times A$
\end{myprop}
\begin{myprop}[Coproducts are neutral wrt Initial Objects]
	$A + 0 \cong A$
\end{myprop}

\begin{myprop}[Distrbutive Law]
In every category that has products and coproducts there is a morphism $A \times B + A \times C \to A \times (B + C)$, iff this morphism is an isomorphism, i.e. $A \times B + A \times C \cong A \times (B + C)$, the underlying category is called \emph{distributive}.
\end{myprop}

\begin{myprop}[Products in $\catSET$]
	$\catSET$ has all products. 
	A binary product in $\catSET$ is given by the well-known cartesian product $A \times B := \{ (a,b) \mid a \in A, b \in B\}$ for $A$ and $B$ being sets.
\end{myprop}
\begin{myprop}[Products in $\catSET$]
	$\catSET$ has all coproducts. 
	A binary coproduct in $\catSET$ is given by the well-known disjoint union $A \uplus B := \{ (i,x) \mid (x \in A \wedge i = 1) \vee (x \in B \wedge i = 2) \}$ for $A$ and $B$ being sets. 
\end{myprop}

\begin{myprop}[Coproduct in $\catINCL$]
	The binary coproduct of $A$ and $B$ in $\catINCL$ is given by their union $A \cup B$.
\end{myprop}

\begin{myprop}[Product in $\catINCL$]
	The binary product of $A$ and $B$ in $\catINCL$ is given by their intersection $A \cap B$.
\end{myprop}

\subsection{Equalizers and Coequalizers}


\begin{mydef}[Equalizer and Coequalizer]
	\[
	\xymatrix{
		Eq(a,b) \ar[r]^{\pi} & A \ar@<1ex>[r]^{a} \ar@<-1ex>[r]_{b} & B & & A \ar@<1ex>[r]^{a} \ar@<-1ex>[r]_{b} & B \ar[r]^{\kappa} \ar@{-->}[dr]_{c} & Ceq(a,b \ar@{.>}[d]^{u!}) \\
		C \ar@{-->}[ur]_{c} \ar@{.>}[u]^{u!} & & & & & & C
	}
	\]
\end{mydef}

\subsection{Pullbacks and Pushouts}


\begin{mydef}[Pullback and Pushout]\label{def:pullbacksAndPushout}
	Given a span of morphisms in a category $\cat{C}$.

	\[
	\xymatrix{
		D \ar@{-->}@/^1pc/[drr]^{g} \ar@{-->}@/_1pc/[ddr]_{f} \ar@{.>}[dr]|{\langle f,g\rangle} & & & & & & & \\
		& A \times_{(a,b)} B \ar[r]^{\pi_B} \ar[d]_{\pi_A}  & B \ar[d]^{b} & & C  \ar[r]^{b} \ar[d]_{a} & B \ar[d]^{\kappa_B} \ar@{-->}@/^1pc/[ddr]^{g} & \\
		& A \ar[r]_{a} & C & & A \ar@{-->}@/_1pc/[drr]_{f} \ar[r]_{\kappa_A} & A +_{(a,b)} B \ar@{.>}[dr]|{[f,g]} & \\
		& & & & & &  D
	}
	\]
	\end{mydef}

\subsection{General (Co-)Limits}

A diagram can be interpreted as a selection of some of the objects and morphisms in $\cat{C}$.
A a cone is a distinguished object together with a family of \emph{outgoing} morphisms that commute with all morphisms in the selection (``sitting under'') and a cocone is a distinguished object together with a family of \emph{incoming} morphisms commuting with all morphisms in the selection (``sitting over'').
The two types of universal construction are limits, i.e. generalized meets (``biggest cone below''), and colimits, i.e. generalized joins (``smallest cocone above''):

\begin{mydef}[Diagram]
	Let $\cat{C}$ be a category. 
	A diagram over $\cat{C}$ is given by a small category $\cat{I}_\mathbf{D}$ and functor $\mathbf{D}: \cat{I}_\mathbf{D} \to \cat{C}$.
\end{mydef}


\begin{mydef}[(Co-)Cone]
	Let $\cat{C}$ be a category and $\mathbf{D}: \cat{I}_\mathbf{D} \to \cat{C}$ be a diagram.
	For an object $C \in \catObjects{\cat{C}}$ we can define a functor $\Delta_C: \cat{I}_\mathbf{D} \to \cat{C}$ that maps every object $I \in \cat{I}_\mathbf{D}$ to $C$ and every arrow $\arrow{I}{i}{I'} \in \cat{I}_\delta(I,I')$ to $id_C$ for $I, I' \in \catObjects{\cat{I}_\mathbf{D}}$.
	A \emph{cone} $(C, \pi: \Delta_C \Rrightarrow \delta)$ is an object together with a natural transformation $\pi$, i. the selection induced by $\delta$.
	A \emph{cocone}$(C, \kappa: \delta \Rrightarrow \Delta_c)$ is the dual notion.
\end{mydef}

\begin{mydef}[(Co-)Limit]
	Let $\cat{C}$ be a category and $\mathbf{D}: \cat{I}_\mathbf{D} \to \cat{C}$ be a diagram.
	
	A \emph{limit} is a cone $(L, \pi: \Delta_L \Rrightarrow \mathbf{D}$ such that for all cones $(X, \xi: \Delta_X \Rrightarrow \mathbf{D}$ there exists a unique morphism $\arrow{X}{u!}{L}$ such that $\pi_I \circ u! = \xi_I$ for all $I \in \cat{I}_\mathbf{D}$.
	The natural transformation $\pi$ is called \emph{projections}.
	
	A \emph{limit} is a cocone $(C, \kappa: \mathbf{D} \Rrightarrow \Delta_C$ such that for all cocones $(X, \xi: \mathbf{D} \Rrightarrow \Delta_X$ there exists a unique morphism $\arrow{C}{u!}{X}$ such that $u! \circ \kappa_I = \xi_I$ for all $I \in \cat{I}_\mathbf{D}$.
	The natural transformation $\kappa$ is called \emph{injections}.
	
	Both notions are visualized in the diagrams below.
	\[
	\xymatrix{
		& X \ar@{-->}@/_1pc/[ddl]_{\xi_I} \ar@{-->}@/^1pc/[ddr]^{\xi{I'}} \ar@{.>}[d]|{u!} & & &
		\mathbf{D}(I) \ar@/_0.5pc/[dr]_{\kappa_I} \ar@{-->}@/_1pc/[ddr]_{\xi_I} \ar[rr]^{\mathbf{D}(i)} & & \mathbf{D}(I') \ar@/^0.5pc/[dl]^{\kappa_I} \ar@{-->}@/^1pc/[ddl]^{\xi_I}\\
		& L \ar@/_0.5pc/[dl]_{\pi_I} \ar@/^0.5pc/[dr]^{\pi_{I'}} & & 
		& & C \ar@{.>}[d]|{u!} & \\
		\mathbf{D}(I) \ar[rr]_{\mathbf{D}(i)} & & \mathbf{D}(I') & &
		& X &
	}
	\]
\end{mydef}


\begin{myprop}[(Co-)Limits are abstract]
	For $\cat{C}$ being a category and $\mathbf{D}: \cat{I}_\mathbf{D} \to \cat{C}$ being a diagram.
	When $(L, \pi: \Delta_L \Rrightarrow \mathbf{D}$ and $(L', \pi': \Delta_{L'} \Rrightarrow \mathbf{D}$ are two limits then $L$ and $L'$ are isomorphic, written $L \cong L'$, i.e. there exists a pair of morphisms $u: L \to L'$ and $u^{-1}: L' \to L$ such that $u^{-1} \circ u = id_L$ and $u \circ u^{-1} = id_{L'}$.
	Dually this holds for colimits as well.
\end{myprop}
\subsection{Initial Objects}

\subsection{Coproducts}

\subsection{Coqualizers}

\subsection{Pushouts}

\subsection{General Colimits}


\section{Adjunctions}
\label{sec:adjunctions}

\input{sections/adjunctions}

\section{Algebras and Monads}
\label{sec:algebras}

\input{sections/algebras}

\section{Partial Arrows and Span Categories}
\label{sec:spans}

\input{sections/spans}


\section{Constructions}
\label{sec:constructions}





\subsection{Constructive Category Theory}

Colored Sets and Graphs, Hypercats as Functor Categories

Inj as a subcategory (epi-refl subcategories)

Incl as a skeleton

Par, Rel, Mult as Kleisli Categories



\section{Toposes}
\label{sec:topoi}

\input{sections/topoi}

\section{Fibered Category Theory}
\label{sec:fibrations}

% TODO Def SLICE and CO-SLICE and ARROW category

\mkDefinition{Fibre Category}{def:fibre}{
Let $\cat{E}$ and $\cat{B}$ be two categories and $\arrow{\cat{E}}{P}{\cat{B}}$ be a functor.
For any object $I \in \catObjects{\cat{B}}$ we define the fibre $\cat{E}_I = \inverse{P}(I)$ over $I$ to be a category, which has 
\catExplanation{All objects $X \in \catObjects{\cat{E}}$ for which $P(X) = I$.}
{All morphisms $\arrow{X}{f}{Y}$ for which $P(f) = \identity{I}$.}
{Retained from composition in $\cat{E}$.}
{Retained from identities in $\cat{E}$.}
}

\mkDefinition{Cartesian and Opcartesian Morphisms}{def:cartesian}{
Let $\cat{E}$ and $\cat{B}$ be two categories and $\arrow{\cat{E}}{P}{\cat{B}}$ be a functor.
\begin{enumerate}
\item An arrow $\arrow{X}{f}{Y}$ with $P(X) = \arrow{I}{u}{J}$ is called \emph{cartesian} over $u$ iff. for any arrow $\arrow{Z}{g}{Y}$ where there exist $\arrow{P(Z)}{w}{I}$ with $\compose{w}{u} = P(g)$ there exists a unique $\arrow{Z}{h}{X}$ such that $\compose{h}{f} = g$ and $P(h) = w$.
\begin{equation}
\label{eq:cartesian}
\tag{Cartesian Arrow}
\xymatrix @C= 5em {
\cat{E} \ar[ddd]_{P}  & \textcolor{blue}{Z} \ar@[blue][dr]_{\textcolor{blue}{g}} \ar@[red]@{-->}[r]^{\textcolor{red}{h!}} & X \ar[d]^f \\
  & & Y \\
  &  \textcolor{blue}{P(Z)} \ar@[blue][r]^{\textcolor{blue}{w}\textcolor{red}{= P(h)}} \ar@[blue][dr]_{\textcolor{blue}{P(g)}} & I  \ar[d]^u \\
\cat{B} & & J
}
\end{equation}
\item  An arrow $\arrow{X}{f}{Y}$ with $P(X) = \arrow{I}{u}{J}$ is called \emph{cartesian} over $u$ iff. for any arrow $\arrow{Z}{g}{Y}$ iff. for any arrow $\arrow{X}{g}{Z}$ where there exists $\arrow{J}{w}{P(Z)}$ with $\compose{u}{w} = P(g)$ there exists a unique $\arrow{X}{h}{Z}$ such that $\compose{f}{g} = h$ and $P(h) = w$.
\begin{equation}
\label{eq:opcartesian}
\tag{Opcartesian Arrow}
\xymatrix @C= 5em {
	\cat{E} \ar[ddd]_{P}  & X \ar[d]^f \ar@[blue][r]^{\textcolor{blue}{g}} &  \textcolor{blue}{Z}  \\ 
	& Y \ar@[red]@{-->}[ur]_{\textcolor{red}{h!}} &  \\
	&  I  \ar[d]_u  \ar@[blue][r]^{\textcolor{blue}{P(g)}} & \textcolor{blue}{P(Z)}    \\
	\cat{B} & J \ar@[blue][ur]_{\textcolor{blue}{w}\textcolor{red}{= P(h)}} & 
}
\end{equation}
\end{enumerate}
A (op)cartesian arrow $f$ is called \emph{weakly (op-)cartesian} if the above conditions only hold for $g$'s being identities $\identity{X}$.
}

\mkProposition{Cartesian Liftings are unique up to isomorphisms}{prop:cartesianLiftingsAbstract}{
Let $\cat{E}$ and $\cat{B}$ be two categories and $\arrow{\cat{E}}{P}{\cat{B}}$ be a functor.
When $\arrow{X}{f}{Y}$ and $\arrow{X'}{f'}{Y'}$ are cartesian over the same map $\arrow{I}{u}{J} \in \catArrows{\cat{B}}$ then there exists a unique isomorphism $\arrow{X}{\isomorphic}{X'}$.
}

\mkDefinition{Fibrations and Opfibrations}{def:fibrations}{
A functor $\arrow{\cat{E}}{P}{\cat{B}}$ is called a \emph{fibration} iff. for every $Y \in \catObjects{\cat{E}}$ and $\arrow{I}{u}{P(Y)} \in \catArrows{\cat{B}}$ there exists a cartesian arrow $\arrow{X}{f}{Y}$ such that $P(f) = u$.\par
Dually, $P$ is called an \emph{opfibration} iff. for every $X \in \catObjects{E}$ and $\arrow{P(X)}{u}{J} \in \catArrows{\cat{B}}$ there exists a an opcartesian arrow $\arrow{X}{f}{Y}$ such that $P(f) = u$.
}

\mkProposition{Codomain Fibrations}{prop:codomainFibration}{
Let $\cat{C}$ be a category and $\arrow{\catArrows{\cat{C}}}{cod}{\cat{C}}$ be the codomain functor.
\begin{enumerate}
\item The fibre category $\cat{C}_I$ over $I \in \catObjects{\cat{C}}$ is equal to the slice category $\sliceCat{\cat{C}}{I}$.
\item Cartesian morphisms in $\catArrows{\cat{C}}$ are exactly the pullback squares in $\cat{C}$.
\item The functor $cod$ is fibration if and only if $\cat{C}$ has pullbacks. 
\end{enumerate}
}

\mkTheorem{Grothendieck Construction}{thm:grothendieck}{
Let $\cat{E}$ and $\cat{B}$ be two categories and $\arrow{\cat{E}}{P}{\cat{B}}$ be a functor.
For any object $B \in \catObjects{\cat{B}}$ there is an equivalence of 2-categories
\begin{equation}
	\mathbf{Fib}(B) \cong \left[\dual{B}, \catCAT \right]
\end{equation}
The functor going from right to left in this equivalence if called. the \emph{Grothendieck construction} (The reconstruction of a fibration from a pseudofunctor) and is denoted
\begin{equation}
 \int : \left[\dual{B}, \catCAT \right] \to \sliceCat{\catCAT}{B}
\end{equation}
}



\section{Adhesive Categories}
\label{sec:adhesive}


\begin{mydef}[Adhesive Category]
	A category $\cat{C}$ is called \emph{adhesive} iff.
	\begin{itemize}
		\item It has all pullbacks.
		\item It has pushouts along monomorphism which are \emph{vanKampen (VK)} squares.
	\end{itemize}
	A pushout square $D \overset{f}{\leftarrow} A \overset{a}{\leftarrow} C \overset{b}{\to} B \overset{g}{\to} D$ is a \emph{VK}-square iff. for every cube where (1.) is at the bottom and the back faces are given by pullbacks the top face is a pushout iff. the front faces are pullbacks.
	\[
	% Van Kampen square
	\xymatrix{
		& &&  & C' \ar[dr]^(0.3){b'} \ar[dl]_(0.3){a'} \ar[d]^{r} &  \\ 
		C \ar[d]_{a} \ar[r]^{b} \ar@{}[dr]|{(1.)}& B \ar[d]^{g} && A' \ar[d]_{p} \ar[dr]_(0.7){f'} & C \ar[dl]|{\hole}_(0.3){a} \ar[dr]|{\hole}^(0.3){b} & B' \ar[dl]^(0.7){g'}  \ar[d]^{q}\\ 
		A \ar[r]_{f} & D && A \ar[dr]_(0.6){f} & D' \ar[d]_{s} & B \ar[dl]^(0.6){g} \\
		&  && & D &
	}
	\]
\end{mydef}


\begin{mydef}[Adhesive HLR category]
	A category $\cat{C}$ is called \emph{adhesive HLR} iff. for $\mathcal{M}$ being a class of monorphisms in $\cat{C}$ that is closed under isomorphisms, composition and decomposition.
	\begin{itemize}
		\item $\cat{C}$ has pullbacks and pushouts along $\mathcal{M}$-morphisms.
		\item Pushouts along $\mathcal{M}$-morphisms are \emph{van Kampen} squares.
	\end{itemize}
\end{mydef}

\begin{myprop}[Pushouts along $\mathcal{M}$ are pullback squares]
...
\end{myprop}
\begin{proof}
Also given in \cite[Lemma 4]{LackSobocinski2004}.
\end{proof}

\begin{myprop}[Pushouts complements of $\mathcal{M}$-morphisms are unique up to isomorphism if they exist]
	...
\end{myprop}
\begin{proof}
	Also given in \cite[Lemma 15]{LackSobocinski2004}.
\end{proof}

\begin{myprop}[Pushouts-pullback decomposition]
	...
\end{myprop}
\begin{proof}
	Also given in \cite[Lemma 16]{LackSobocinski2004}.
\end{proof}

\section{Internal Category Theory}
\label{sec:internalCT}

% TODO working internal to a category with pullbacks

\section{Regular Categories and Allegories}
\label{sec:allegories}

\section{Higher Order Category Theory}
\label{sec:HO}

% TODO notion of enriched


\label{sect:bib}
\bibliographystyle{plain}
%\bibliographystyle{alpha}
%\bibliographystyle{unsrt}
%\bibliographystyle{abbrv}
\bibliography{../../git/literature/library}

%------------------------------------------------------------------------------


%------------------------------------------------------------------------------
% Index
%\printindex

%------------------------------------------------------------------------------
\end{document}

